\documentclass[11pt]{article}

\usepackage{setspace}
\usepackage[activate={true,nocompatibility},
            final,
            tracking=true,
            kerning=true,
            spacing=true,
            factor=1100,
            stretch=10,
            shrink=10]{microtype}

\microtypecontext{spacing=nonfrench}

\doublespacing

\setlength\parskip{1ex}
\setlength\parindent{1em}

\begin{document}

\noindent
\textbf{Dictionaries and Lists}\\*
Isaac Dudney, Jeremy Hansen, Aida Popa

We'll be comparing and showing runtimes of list and dictionary functions.
It'll be split into 3 parts: dictionaries, lists, and the common functions between them.
The dictionary and list sections will simply report runtimes of functions on various inputs.
The common functions will compare the two.

\noindent
\textbf{Lists}

%todo

\noindent
\textbf{Dictionaries}

Dictionaries have 4 functions that are not shared with lists.

At 200,000 key/value pairs, the slowest operation was grabbing the items (\texttt{items}) at 0.31 seconds, followed by grabbing all values (\texttt{values}), which took 0.21 seconds.
Grabbing just the keys (\texttt{keys}), however, took only 0.069 seconds, which is an order of magnitude faster than either other operation; it seems grabbing keys is much easier than grabbing their values, suggesting that the only way to grab a value is through its key (or else accessing keys and values should take around the same time).
The fastest function was the \texttt{get} function, at only 0.052 seconds, faster by a small margin than grabbing all keys.
The \texttt{get} function provides a specific key/value pair, grabbed by key.

\noindent
\textbf{Common}



\end{document}
